\documentclass[11pt, oneside, a4paper]{article}
\usepackage[cp1251]{inputenc} % encoding
\usepackage[english]{babel} % English hyphenation
\usepackage{graphicx}          % to include the graphics
\usepackage{cite}              % for correct references design 
\usepackage{pavt-en}                                

\begin{document}

% \title - paper title
% \authors - authors list

\title{Parallel algorithm for solving initial boundary value problem for time-fractional Korteweg-de Vries equation}

\authors{M.S.~Grebelnick\superscript{1}, S.Yu.~Lukashchuk\superscript{1}}
\organizations{Ufa State Aviation Technical University\superscript{1}}

% abstract is contained in the  abstract environment 
\begin{abstract}
We consider the problem of numerical solution to the initial boundary value problem for a fractional generalization of the Korteweg-de Vries equation with the Riemann-Liouville time-fractional derivative. The presence of a fractional derivative leads to a linear increasing computational cost as the number of time step increases. As a result, solving even a relatively simple one-dimensional problem requires very considerable computational resources, which leads to the necessity for developing parallel algorithms for its solving. We present a numerical scheme for solving the problem and an algorithm for its parallelization, as well as an estimation of its efficiency. The proposed approach is justified by the results of computational experiments performed on computing systems with shared memory.
\end{abstract}

\keywords{fractional derivative, fractional differential equation, numerical scheme, parallel algorithm, efficiency}

% \section{title} - the header is the first level section
% \subsection{title} - the header is the second level section
% \subsubsection{title} - the header is the third level section
% Do not use the nesting level of more than three headlines!
% Each paragraph of the paper begins with \par command or empty line

\section{Introduction}
This document contains guidelines for \LaTeX users 
(you can also use guidelines for MS~Word users) and can be considered as a template. 

\section{Paper preparation}
Pages of the presented paper should have size
$297 \times 210$~mm (A4 format). All margins should have the same size of 25~mm.

Numbering, forced page breaks and footers are not allowed.

Times New Roman (or equivalent) is used as a font of the paper. 
Paragraphs are typed by 11~pt font size with width alignment, 
single line spacing, and automatic hyphenation. Using bold style in the text is 
undesirable. Paragraphs are not separated by intervals and begin with a 7~mm indented line.

\subsection{Headings}
For the title of the paper bold Times New Roman font with 16~pt size and center alignment should be used. 
Title is separated by one blank line from all the text above.

List of authors appears in 12~pt Times New Roman with center alignment and is 
separated from the title by a single blank line with 16~pt size. Authors are listed 
through commas, initials are written before the last name. 

Organization has 12~pt Times New Roman font with center alignment and is separated 
from the list of authors by one blank line of 6~pt size. Please, specify the official names (not abbreviations) of organizations. 
If authors work in different organizations, two or more organizations can be listed through commas. 
In this case author's affiliation is identified by the footnote.

Header of a section is made without a break from the first paragraph with left-alignment. 
Above and below the header there is a single blank line separation. 
Examples of titles and styles are shown in table ~\ref{table1}.

\subsection{Figures and tables}
All figures and tables should have a caption, typed by 10~pt font with top and bottom 6~pt 
separation and central alignment. Caption to a table is placed above the table, 
caption to a figure~--- under the figure. 

\textbf{Assertions}, \textbf{lemmas} and \textbf{theorems} should be formatted as separate paragraphs and 
should be numbered in order of appearance starting from number one, for example, \textbf{Theorem~1}.

\begin{table}[!ht]
	\caption{Heading examples}
	\label{table1}
	\begin{center}
	\begin{tabular}{|l|l|}
		\hline
		\multicolumn{1}{|c|}{\bf Type of the header} &
		\multicolumn{1}{c|}{\bf Font size and style} \\
		\hline
		\LARGE\bf A title of the paper & 16~pt, bold \\
		\hline
		\Large\bf First-level header & 14~pt, bold \\
		\hline
		\large\bf Second-level header& 12~pt, bold \\
		\hline
		\large\it Third-level header & 12~pt, italic \\
		\hline
	\end{tabular}
	\end{center}
\end{table}

\subsection{Formulae and source code}
Formulae are typed with standard \LaTeX features.

Source code should be formatted with Courier New font(or equivalent), 
of 10~pt size (see an example on Fig.~\ref{listing-sort}). 
Bold and italic font is allowed in the source code.

\begin{figure}[h]
\begin{lstlisting}[language=c]
void selectionsort(ap::real_1d_array& arr, const int& n)
{
    int i;
    int j;
    int k;
    double m;
    for(i = 1; i <= n; i++) {
        m = arr(i-1);
        k = i;
        for(j = i; j <= n; j++) {
            if( m>arr(j-1) ) {
                m = arr(j-1);
                k = j;
            }
        }
        arr(k-1) = arr(i-1);
        arr(i-1) = m;
    }
}
\end{lstlisting}
\caption{Selection sort algorithm}
\label{listing-sort}
\end{figure}

\subsection{Footnotes and cross-references}
Footnotes are placed at the bottom of the page and numbered with Arabic numerals 
\footnote{Footnote numeration starts from the beginning on every new page.}. 
Acknowledgements for the financial support should be performed 
as a footnote to the title of the paper and 
denoted by * symbol. 

Cross-references should be placed in square brackets and listed in ascending order, separated 
by commas or dashes, for example:~\cite{stonebraker},
\cite{stonebraker,  amit, cadez}, \cite{yao, amit,cadez}. 

Reference list is numbered in Arabic numerals with font of 11~pt and with left-alignment.

\begin{biblio}
\bibitem{samko}
Samko~S., Kilbas~A., Marichev~O. Integraly i proizvodnye drobnogo poriadka i nekotorye ikh prilozheniia. Minsk: Nauka i tekhnika, 1987. 688~p.
\bibitem{trujillo}
Kilbas~A., Srivastava~H., Trujillo~J. Theory and applications of fractional differential equations. North-Holland Mathematics Studies 204, 2006. 523~p.
\bibitem{podlub}
Podlubny~I. Fractional differential equations. Academic press, 1999. 340~p.
\bibitem{samarsk}
Samarskii~A., Nikolaev~E. Metody resheniia setochnykh uravnenii. Moskva: Nauka, 1978. 588~p.
\end{biblio}
\end{document}