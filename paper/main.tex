\documentclass[11pt, oneside, a4paper]{article}
\usepackage[cp1251]{inputenc} % encoding
\usepackage[english]{babel} % English hyphenation
\usepackage{graphicx}          % to include the graphics
\usepackage{cite}              % for correct references design 
\usepackage{pavt-en}                

\newcommand {\partt} {\frac{\partial}{\partial t}}       
\newcommand {\dt} {\Delta t}
\usepackage{amsmath}

\begin{document}

% \title - paper title
% \authors - authors list

\title{Parallel algorithm for solving initial boundary value problem for time-fractional Korteweg-de Vries equation}

\authors{M.S.~Grebelnick\superscript{1}, S.Yu.~Lukashchuk\superscript{1}}
\organizations{Ufa State Aviation Technical University\superscript{1}}

% abstract is contained in the  abstract environment 
\begin{abstract}
We consider the problem of numerical solution to the initial boundary value problem for a fractional generalization of the Korteweg-de Vries equation with the Riemann-Liouville time-fractional derivative. The presence of a fractional derivative leads to a linear increasing computational cost as the number of time step increases. As a result, solving even a relatively simple one-dimensional problem requires very considerable computational resources, which leads to the necessity for developing parallel algorithms for its solving. We present a numerical scheme for solving the problem and an algorithm for its parallelization, as well as an estimation of its efficiency. The proposed approach is justified by the results of computational experiments performed on computing systems with shared memory.
\end{abstract}

\keywords{fractional derivative, fractional differential equation, numerical scheme, parallel algorithm, efficiency}

% \section{title} - the header is the first level section
% \subsection{title} - the header is the second level section
% \subsubsection{title} - the header is the third level section
% Do not use the nesting level of more than three headlines!
% Each paragraph of the paper begins with \par command or empty line

\section{Introduction}
\begin{definition}
	Riemann--Liouville integral of the order $1 - \alpha$, from $0$ to $t$ of the function $U(x,t)$:
	\begin{equation}
	{}_{0}J_{t}^{1-\alpha} U(x,t) = \frac{1}{\Gamma(1-\alpha)} \int_{0}^{t} \frac{U(x,\tau)}{(t-\tau)^{\alpha}}d\tau
	\end{equation}
\end{definition}
\begin{definition}
	Riemann--Liouville fractional derivative of the order $\alpha$, from $0$ to $t$ of the function $U(x,t)$:
	\begin{equation}
	{}_{0}D_{t}^{\alpha} U(x,t) = \frac{1}{\Gamma(1-\alpha)} \partt \int_{0}^{t} \frac{U(x,\tau)}{(t-\tau)^{\alpha}}d\tau
	\end{equation}
\end{definition}
\begin{definition}
	Gamma function:
	\begin{equation}
	\Gamma(z)=\int_{0}^{+\infty}t^{z-1} e^{-t} dt
	\end{equation}
\end{definition}
\begin{definition}
	Grunwald--Letnikov approximation for the fractional derivative:
	\begin{equation}
	{}_{a}D_{t}^{\alpha}U(x,t) = \lim_{h \to 0} \frac{1}{\dt^{\alpha}} \sum_{j = 0}^{\left[\frac{t-a}{h} \right]} (-1)^{j} \binom{\alpha}{j} U(x, t - jh)
	\end{equation}
\end{definition}

\section{Problem statement}
The following initial-boundary problem is being solved:
\begin{equation}
	{}_{0}D_{t}^{\alpha}U(x,t) + U(x,t)\cdot {}_{0}J^{1-\alpha}_{t}U_{x}(x,t)=a \cdot U_{xxx}(x,t),~\alpha \in (0, 1),~a = const,
\end{equation}
\begin{equation}
	{}_{0}J^{1-\alpha}_{t}U(x,t) \bigg{|}_{t=0} = f(x),
\end{equation}
\begin{equation}
U_{x}(x,t) \bigg{|}_{x=0}=0,~U_{x}(x,t) \bigg{|}_{x=1}=0,~U_{xx}(x,t) \bigg{|}_{x=1}=0.
\end{equation}

Due to infinity in the initial conditions at $t=0$, the following substitution is made:
\begin{equation}
	U(x,t) = U(x;s) + V(x,t),
\end{equation}
where $U(x;s)$ is:
\begin{equation}
	U(x;s) = \frac{s^{\alpha-1} \cdot f(x)}{\Gamma(\alpha)}.
\end{equation}

\section{Numerical scheme}

\section{Parallel algorithm implementation}

\section{Computational experiment}

\section{Conclusion}

\begin{biblio}
\bibitem{samko}
Samko~S., Kilbas~A., Marichev~O. Integraly i proizvodnye drobnogo poriadka i nekotorye ikh prilozheniia. Minsk: Nauka i tekhnika, 1987. 688~p.
\bibitem{trujillo}
Kilbas~A., Srivastava~H., Trujillo~J. Theory and applications of fractional differential equations. North-Holland Mathematics Studies 204, 2006. 523~p.
\bibitem{podlub}
Podlubny~I. Fractional differential equations. Academic press, 1999. 340~p.
\bibitem{samarsk}
Samarskii~A., Nikolaev~E. Metody resheniia setochnykh uravnenii. Moskva: Nauka, 1978. 588~p.
\end{biblio}
\end{document}